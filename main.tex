\documentclass[french]{article}
\usepackage[french]{babel}
\usepackage{graphicx}
\usepackage{epstopdf}
\usepackage{amsmath}
\usepackage{amssymb}
\usepackage{tikz}
\usepackage[utf8]{inputenc}
\usepackage[T1]{fontenc}

\graphicspath{{./plots/}}

\title{TP Hydro}
\author{Lucas Gautheron}
\date{}

\begin{document}

\maketitle

\section{Effet Janssen dans une colonne granulaire statique ou quasi-statique}

\subsection{Expérience}

La première expérience consiste à mesurer la pression à la base d'une colonne granulaire quasi-statique. Une colonne cylindrique d'une hauteur d'environ 40 cm et 36 mm de diamètre intérieur est disposée verticalement comme indiqué sur la figure \ref{fig:montage_janseen}. 

\begin{figure}
    \centering
    % GNUPLOT: LaTeX picture with Postscript
\begingroup
  \makeatletter
  \providecommand\color[2][]{%
    \GenericError{(gnuplot) \space\space\space\@spaces}{%
      Package color not loaded in conjunction with
      terminal option `colourtext'%
    }{See the gnuplot documentation for explanation.%
    }{Either use 'blacktext' in gnuplot or load the package
      color.sty in LaTeX.}%
    \renewcommand\color[2][]{}%
  }%
  \providecommand\includegraphics[2][]{%
    \GenericError{(gnuplot) \space\space\space\@spaces}{%
      Package graphicx or graphics not loaded%
    }{See the gnuplot documentation for explanation.%
    }{The gnuplot epslatex terminal needs graphicx.sty or graphics.sty.}%
    \renewcommand\includegraphics[2][]{}%
  }%
  \providecommand\rotatebox[2]{#2}%
  \@ifundefined{ifGPcolor}{%
    \newif\ifGPcolor
    \GPcolorfalse
  }{}%
  \@ifundefined{ifGPblacktext}{%
    \newif\ifGPblacktext
    \GPblacktexttrue
  }{}%
  % define a \g@addto@macro without @ in the name:
  \let\gplgaddtomacro\g@addto@macro
  % define empty templates for all commands taking text:
  \gdef\gplbacktext{}%
  \gdef\gplfronttext{}%
  \makeatother
  \ifGPblacktext
    % no textcolor at all
    \def\colorrgb#1{}%
    \def\colorgray#1{}%
  \else
    % gray or color?
    \ifGPcolor
      \def\colorrgb#1{\color[rgb]{#1}}%
      \def\colorgray#1{\color[gray]{#1}}%
      \expandafter\def\csname LTw\endcsname{\color{white}}%
      \expandafter\def\csname LTb\endcsname{\color{black}}%
      \expandafter\def\csname LTa\endcsname{\color{black}}%
      \expandafter\def\csname LT0\endcsname{\color[rgb]{1,0,0}}%
      \expandafter\def\csname LT1\endcsname{\color[rgb]{0,1,0}}%
      \expandafter\def\csname LT2\endcsname{\color[rgb]{0,0,1}}%
      \expandafter\def\csname LT3\endcsname{\color[rgb]{1,0,1}}%
      \expandafter\def\csname LT4\endcsname{\color[rgb]{0,1,1}}%
      \expandafter\def\csname LT5\endcsname{\color[rgb]{1,1,0}}%
      \expandafter\def\csname LT6\endcsname{\color[rgb]{0,0,0}}%
      \expandafter\def\csname LT7\endcsname{\color[rgb]{1,0.3,0}}%
      \expandafter\def\csname LT8\endcsname{\color[rgb]{0.5,0.5,0.5}}%
    \else
      % gray
      \def\colorrgb#1{\color{black}}%
      \def\colorgray#1{\color[gray]{#1}}%
      \expandafter\def\csname LTw\endcsname{\color{white}}%
      \expandafter\def\csname LTb\endcsname{\color{black}}%
      \expandafter\def\csname LTa\endcsname{\color{black}}%
      \expandafter\def\csname LT0\endcsname{\color{black}}%
      \expandafter\def\csname LT1\endcsname{\color{black}}%
      \expandafter\def\csname LT2\endcsname{\color{black}}%
      \expandafter\def\csname LT3\endcsname{\color{black}}%
      \expandafter\def\csname LT4\endcsname{\color{black}}%
      \expandafter\def\csname LT5\endcsname{\color{black}}%
      \expandafter\def\csname LT6\endcsname{\color{black}}%
      \expandafter\def\csname LT7\endcsname{\color{black}}%
      \expandafter\def\csname LT8\endcsname{\color{black}}%
    \fi
  \fi
    \setlength{\unitlength}{0.0500bp}%
    \ifx\gptboxheight\undefined%
      \newlength{\gptboxheight}%
      \newlength{\gptboxwidth}%
      \newsavebox{\gptboxtext}%
    \fi%
    \setlength{\fboxrule}{0.5pt}%
    \setlength{\fboxsep}{1pt}%
\begin{picture}(7200.00,5040.00)%
      \csname LTb\endcsname%
      \put(3600,4820){\makebox(0,0){\strut{}Auto-layout of stacked plots}}%
      \put(3600,4700){\makebox(0,0){\strut{}}}%
    \gplgaddtomacro\gplbacktext{%
      \csname LTb\endcsname%
      \put(264,1512){\makebox(0,0)[r]{\strut{}$0$}}%
      \put(264,2167){\makebox(0,0)[r]{\strut{}$0.1$}}%
      \put(264,2822){\makebox(0,0)[r]{\strut{}$0.2$}}%
      \put(264,3477){\makebox(0,0)[r]{\strut{}$0.3$}}%
      \put(264,4132){\makebox(0,0)[r]{\strut{}$0.4$}}%
      \put(264,4787){\makebox(0,0)[r]{\strut{}$0.5$}}%
      \put(396,1292){\makebox(0,0){\strut{}$0$}}%
      \put(1677,1292){\makebox(0,0){\strut{}$0.5$}}%
      \put(2959,1292){\makebox(0,0){\strut{}$1$}}%
      \put(4240,1292){\makebox(0,0){\strut{}$1.5$}}%
      \put(5522,1292){\makebox(0,0){\strut{}$2$}}%
      \put(6803,1292){\makebox(0,0){\strut{}$2.5$}}%
    }%
    \gplgaddtomacro\gplfronttext{%
      \csname LTb\endcsname%
      \put(-374,3149){\rotatebox{-270}{\makebox(0,0){\strut{}m (kg)}}}%
      \put(3599,962){\makebox(0,0){\strut{}U (V)}}%
      \csname LTb\endcsname%
      \put(5816,4614){\makebox(0,0)[r]{\strut{}Mesures}}%
      \csname LTb\endcsname%
      \put(5816,4394){\makebox(0,0)[r]{\strut{}fit affine}}%
    }%
    \gplgaddtomacro\gplbacktext{%
      \csname LTb\endcsname%
      \put(264,504){\makebox(0,0)[r]{\strut{}$-0.002$}}%
      \put(264,630){\makebox(0,0)[r]{\strut{}$-0.0015$}}%
      \put(264,756){\makebox(0,0)[r]{\strut{}$-0.001$}}%
      \put(264,882){\makebox(0,0)[r]{\strut{}$-0.0005$}}%
      \put(264,1008){\makebox(0,0)[r]{\strut{}$0$}}%
      \put(264,1134){\makebox(0,0)[r]{\strut{}$0.0005$}}%
      \put(264,1260){\makebox(0,0)[r]{\strut{}$0.001$}}%
      \put(264,1386){\makebox(0,0)[r]{\strut{}$0.0015$}}%
      \put(264,1512){\makebox(0,0)[r]{\strut{}$0.002$}}%
      \put(396,284){\makebox(0,0){\strut{}$0$}}%
      \put(1677,284){\makebox(0,0){\strut{}$0.5$}}%
      \put(2959,284){\makebox(0,0){\strut{}$1$}}%
      \put(4240,284){\makebox(0,0){\strut{}$1.5$}}%
      \put(5522,284){\makebox(0,0){\strut{}$2$}}%
      \put(6803,284){\makebox(0,0){\strut{}$2.5$}}%
    }%
    \gplgaddtomacro\gplfronttext{%
      \csname LTb\endcsname%
      \put(-902,1008){\rotatebox{-270}{\makebox(0,0){\strut{}m (kg)}}}%
      \put(3599,-46){\makebox(0,0){\strut{}U (V)}}%
      \csname LTb\endcsname%
      \put(5816,1339){\makebox(0,0)[r]{\strut{}Résidus}}%
    }%
    \gplbacktext
    \put(0,0){\includegraphics{fit_capteur}}%
    \gplfronttext
  \end{picture}%
\endgroup

    \caption{Caption}
    \label{fig:my_label}
\end{figure}


\subsection{Résultats}

\section{Écoulement granulaire dans une colonne et loi de Beverloo}

\subsection{Expérience}
\subsection{Résultats}

\end{document}
